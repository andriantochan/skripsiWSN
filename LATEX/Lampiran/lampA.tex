%versi 3 (18-12-2016)
\chapter{Kode Program}
\label{lamp:A}

%terdapat 2 cara untuk memasukkan kode program
% 1. menggunakan perintah \lstinputlisting (kode program ditempatkan di folder yang sama dengan file ini)
% 2. menggunakan environment lstlisting (kode program dituliskan di dalam file ini)
% Perhatikan contoh yang diberikan!!
%
% untuk keduanya, ada parameter yang harus diisi:
% - language: bahasa dari kode program (pilihan: Java, C, C++, PHP, Matlab, C#, HTML, R, Python, SQL, dll)
% - caption: nama file dari kode program yang akan ditampilkan di dokumen akhir
%
% Perhatian: Abaikan warning tentang textasteriskcentered!!
%


%package SensorNode
\lstinputlisting[label={lamp:accelerometer}, language=Java, caption=Accelerometer.java]{./Lampiran/SensorNode/Accelerometer.java} 
\lstinputlisting[label={lamp:sensorManager}, language=Java, caption=SensorManager.java]{./Lampiran/SensorNode/SensorManager.java} 

%package BaseStation
\lstinputlisting[label={baseStation}, language=Java, caption=BaseStation.java]{./Lampiran/BaseStation/BaseStation.java} 

%package Controller
\lstinputlisting[label={lamp:complex}, language=Java, caption=Complex.java]{./Lampiran/Controller/Complex.java}

\lstinputlisting[label={lamp:dataFrequency}, language=Java, caption=DataFrequency.java]{./Lampiran/Controller/DataFrequency.java}

\lstinputlisting[label={lamp:fft}, language=Java, caption=FFT.java]{./Lampiran/Controller/FFT.java}

\lstinputlisting[label={lamp:renderChart}, language=Java, caption=RenderChart.java]{./Lampiran/Controller/RenderChart.java}

\lstinputlisting[label={lamp:sampleData}, language=Java, caption=SampleData.java]{./Lampiran/Controller/SampleData.java}

\lstinputlisting[label={lamp:sensor}, language=Java, caption=Sensor.java]{./Lampiran/Controller/Sensor.java}

\lstinputlisting[label={lamp:startChart}, language=Java, caption=StartChart.java]{./Lampiran/Controller/StartChart.java}

\chapter{KESIMPULAN DAN SARAN}

\section{Kesimpulan}
Berdasarkan hasil penelitian yang dilakukan, diperoleh kesimpulan-kesimpulan sebagai berikut:
\begin{enumerate}
    \item Cara kerja sensor node, sensor accelerometer dan WSN telah berhasil dipelajari
    \item Aplikasi pemantauan getaran gedung menggunakan WSN telah berhasil dibangun.
    \item Topologi yang digunakan yaitu \textit{star} dan \textit{tree} untuk melakukan pengujian tidak berpengaruh terhadap hasil pemantauan.
    \item Gedung yang dilakukan pengujian masih dalam kondisi sehat berdasarkan dengan keadaan gedung, jenis gedung dan juga hasil dari nilai amplitudo ataupun frekuensi.
\end{enumerate}

\section{Saran}
Berdasarkan hasil penelitian yang dilakukan, ada beberapa saran untuk pengembangan aplikasi sebagai berikut:
\begin{enumerate}
    \item Aplikasi ini hanya dibangun untuk melakukan pemantauan getaran, lebih baik apabila aplikasi dikembangkan agar dapat  mengklasifikasi jenis getaran yang dipantau.
    
    \item Aplikasi lebih baik diaplikasikan ke lingkungan big data, supaya akurasi hasil getaran yang ditangkap lebih akurat. 
    
    \item Jumlah sensor node yang digunakan untuk pengujian dapat lebih banyak sehingga lingkungan jangkauan pengujian dapat dilakukan dengan skala yang lebih besar.
\end{enumerate}